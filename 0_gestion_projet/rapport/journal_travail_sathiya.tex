\section{Journal de travail - Sathiya Kirushnapillai}

\subsection{Semaine 1: 20 février - 24 février}
Constitution des groupes et choix du sujet. De nombreux sujets ont été proposés, la plupart ne faisant pas l'unanimité. Finalement, deux idées ont fait été retenues par le groupe: un programme de manipulation de graphes, et en premier lieu, un programme d'édition d'images.

\subsection{Semaine 2: 27 février - 3 mars}
Attribution des sujets de projet. Le projet de programme d'édition d'images a été accepté. Nous avons donc pu commencer la réflexion autour des fonctionnalités et la planification du projet.

\subsection{Semaine 3: 6 mars - 10 mars}
Discussion en groupe. Nous faisons le tri des fonctionnalités indispensables et utiles. Une fois celles-ci fixées, nous établissons le cahier des charges et la planification Gant qui va avec.

\subsection{Semaine 4: 13 mars - 17 mars}
Le professeur nous fourni un retour sur notre cahier des charges et notre planification. Pour plus de clarté, nous remplissons une nouvelle planification dans un format Excel. Nous nous mettons d'accord sur le fait de prendre deux semaines pour étudier la technologie JavaFX que nous utiliserons pour le projet et qu'aucun de nous ne connaît.

\subsection{Semaine 5: 20 mars - 24 mars}
Étude de JavaFX.

\subsection{Semaine 6: 27 mars - 31 mars}
Étude de JavaFX et première réalisation du diagramme de classe.

\subsection{Semaine 7: 3 avril - 7 avril}
Création de la classe Document avec les fonctionnalités suivantes : Ouvrir, sauver et exporter un document. Cependant, le Workspace n'est pas encore disponible pour compléter la classe Document.

\subsection{Semaine 8: 10 avril - 14 avril}
Étude de la sérialisation avec JavaFX. Malheureusement, les classes de JavaFX n'implémentent pas Serializable. Recherche d'une solution afin de combler ce manque pour Canvas, Text, Image, etc.

\subsection{Semaine 9: 24 avril - 28 avril}
Étude de la sérialisation avec JavaFX. Implémentation de la sérialisation sur des classes de base tel que Text et Canvas. 

\subsection{Semaine 10: 1 mai - 5 mai}
Réécriture de la classe Document avec le Workspace enfin disponible. Implémentation concrète de la sérialisation et réalisation de tests avec des classes ajouter dans le Workspace.

Ajout des boutons permettant de créer, ouvrir et sauver un document.

\subsection{Semaine 11: 8 mai - 12 mai} 
Gestion de plusieurs documents à l'aide d'onglets (Adaptation de la classe Document). 

Ajout du bouton permettant l'export d'un document en image PNG.

\subsection{Semaine 12: 15 mai - 19 mai}
Ajout de l'importation d'une image et de sa sérialisation. 

Ajout de l'outil sélection.

Redimensionnement du Workspace à l'aide d'un popup.

Amélioration visuelle de l'interface des outils.

Ajout d'un fond de couleur à la création d'un document

\subsection{Semaine 13: 22 mai - 26 mai}
Il y a eu beaucoup de problèmes avec les transformations de calques(rotation, taille, symétrie, etc). Recherche d'une solution avec Mathieu Monteverde. 

Implémentation du déplacement du rectangle de sélection et de la suppression à l'aide de l'outil sélection.

\subsection{Semaine 14: 29 mai - 2 juin}
Finition du code (Commentaires, bugs, structures) et du rapport.




