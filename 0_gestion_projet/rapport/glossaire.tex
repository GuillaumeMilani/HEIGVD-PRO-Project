\newglossaryentry{workspace}{%
  name={Workspace},
  description={Workspace est une classe de l'application GEMMS représentant la zone de travail de l'utilisateur dans un Document GEMMS. Un objet contient notamment la liste de tous les calques utilisés dans le document}
}
\newglossaryentry{stage}{%
	name={Stage},
	description={Le Stage est le container haut-niveau d'une application JavaFX. Il doit contenir tout le contenu d'une fenêtre}
}

\newglossaryentry{scene}{%
	name={Scene},
	description={La Scene, dans une application JavaFX, est le container pour un Scene Graph. Chaque scène doit avoir un et un seul nœud racine du Scene Graph}
}
\newglossaryentry{scene_graphe}{%
	name={Scene Graph},
	description={Le Scene Graph (graphe de scène en français) est une structure en arbre qui garde une représentation interne des éléments graphiques de l'application}
}
\newglossaryentry{coloradjust}{%
name={ColorAdjust},
description={Effet JavaFX permettant d'ajuster la couleur d'un n\oe ud en modifiant son contraste, saturation, teinte et luminosité.}
}
\newglossaryentry{sepiatone}{%
name={SepiaTone},
description={Effet JavaFX permettant d'affecter à un n\oe ud un effet sépia d'intensité variable.}
}
\newglossaryentry{gaussianblur}{%
name={GaussianBlur},
description={Effet JavaFX permettant d'affecter à un n\oe ud un effet de flou gaussien d'intensité variable.}
}
\newglossaryentry{node}{name={Node},description={La classe Node est une classe de JavaFX. Il s'agit de la classe de base de tout élément présent dans le Scene Graph}}
\newglossaryentry{anchorpane}{name={AnchorPane},description={AnchorPane est une classe de JavaFX permettant (entre autres) de disposer des noeuds selon des coordonnées (x,y)}}
