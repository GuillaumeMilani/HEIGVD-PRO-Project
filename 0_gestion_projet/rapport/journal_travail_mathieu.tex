\section{Journal de travail - Mathieu Monteverde}

\subsection{Semaine 1: 20 février - 24 février}
Constitution des groupes et choix du sujet. De nombreux sujets ont été proposés, la plupart ne faisant pas l'unanimité. Finalement, deux idées ont fait été retenues par le groupe: un programme de manipulation de graphes, et en premier lieu, un programme d'édition d'images.
\subsection{Semaine 2: 27 février - 3 mars}
Attribution des sujets de projet. Le projet de programme d'édition d'images a été accepté. Nous avons donc pu commencer la réflexion autour des fonctionnalités et la planification du projet.
\subsection{Semaine 3: 6 mars - 10 mars}
Discussion en groupe. Nous faisons le tri des fonctionnalités indispensables et utiles. Une fois celles-ci fixées, nous établissons le cahier des charges et la planification Gant qui va avec.
\subsection{Semaine 4: 13 mars - 17 mars}
Le professeur nous fourni un retour sur notre cahier des charges et notre planification. Pour plus de clarté, nous remplissons une nouvelle planification dans un format Excel. Nous nous mettons d'accord sur le fait de prendre deux semaines pour étudier la technologie JavaFX que nous utiliserons pour le projet et qu'aucun de nous ne connaît.
\subsection{Semaine 5: 20 mars - 24 mars}
Étude de JavaFX.
\subsection{Semaine 6: 27 mars - 31 mars}
Étude de JavaFX. Début de la mise en place de la structure de la classe Workspace avec la spécification des méthodes. Nous réalisons également que des difficultés sont à attendre pour la gestion des calques, de la sérialisation et des éléments issus de JavaFX en général.
\subsection{Semaine 7: 3 avril - 7 avril}
Implémentation du Workspace avec insertion et suppression de calques.
\subsection{Semaine 8: 10 avril - 14 avril}
Implémentation de la gestion du Workspace pour le déplacement et le zoom de l'utilisateur dans l'interface. Premières recherches pour la gestion de calques au moyen d'une ListView de JavaFX.
\subsection{Semaine 9: 24 avril - 28 avril}
Ajout des prototypes recherchés en semaine 9 au reste du projet. Le Workspace permet maintenant d'ajouter des calques, de les supprimer, et de se déplacer dans l'interface. 
\subsection{Semaine 10: 1 mai - 5 mai}
Des changements ont eu lieu pour le Workspace. De mon côté, il faut encore améliorer la gestion des calques.
On se rend compte que l'élément ListView de JavaFX, qui pourrait pourtant de fonctionner parfaitement pour l'affichage des calques, ne permet de pas de changer l'ordre d'affichage.
\subsection{Semaine 11: 8 mai - 12 mai} 
Il va falloir trouver une solution pour la gestion des calques. Le module étant cependant fonctionnel, on se charge des autres fonctionnalités. Début de la réalisation du pinceau et de la gomme.
\subsection{Semaine 12: 15 mai - 19 mai}
Remplacement de la ListView de gestion des calques JavaFX par un composant fait main pour pouvoir répondre aux besoins de l'application. Les outils pinceaux, gommes et pipette sont fonctionnels. Beaucoup d'élément ont pu être ajoutés cette semaine. La gestion de la couleur, le paramétrage des outils (taille du pinceau et de la gomme, édition de textes et autres).
\subsection{Semaine 13: 22 mai - 26 mai}
Il y a eu beaucoup de problèmes avec les transformations de calques (rotation, taille, symétrie, etc.) Mais après beaucoup d'efforts, les problèmes ont pu être résolu. Maintenant, il s'agit de peaufiner l'interface (espacement, ergonomie, retour visuels pour l'utilisateur, etc.),
\subsection{Semaine 14: 29 mai - 2 juin}
Finition du rapport et du manuel utilisateur.



