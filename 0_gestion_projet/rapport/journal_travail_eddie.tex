\section{Journal de travail - Edward Ransome}

\subsection{Semaine 1: 20 février - 24 février}
Création du groupe et recherche d'idées. Discussion et proposition des deux principaux sujets au professeur, un éditeur d'images et un programme de manipulation de graphes.
\subsection{Semaine 2: 27 février - 3 mars}
Programme d'édition d'images accepté, début d'élaboration du cahier des charges. Discussion sur le fonctionnement, l'architecture, les fonctionnalités voulues ainsi que des mock-ups d'interface. Travail simultané de tout le groupe.
\subsection{Semaine 3: 6 mars - 10 mars}
Fin de rédaction du cahier des charges, avec mock-ups finaux et une liste des fonctionnalités indispensables et optionnelles. Rendu de celui-ci ainsi qu'un planning du travail sous forme de diagramme de Gant.
\subsection{Semaine 4: 13 mars - 17 mars}
Retour sur le cahier des charges, pas de problèmes majeurs. Planification rédigée sous forme d'un tableau Excel pour faciliter la compréhension et mieux voir le travail effectué à chaque membre du groupe. Début du travail individuel selon le planning. 
\subsection{Semaine 5: 20 mars - 24 mars}
Étude de JavaFX. Tutoriel JavaFX 8 de Code Makery effectué, qui comprends une introduction à Scene Builder, création de fenêtres, effets et autres fonctionnalités de cette librairie. Étude de la documentation Oracle de JavaFX.
\subsection{Semaine 6: 27 mars - 31 mars}
Début de création de l'interface avec Guillaume Milani en se basant sur les mock-ups élaborés pour le cahier des charges.
\subsection{Semaine 7: 3 avril - 7 avril}
Fin de l'interface Scene Builder avec contrôleurs des boutons dans le code ainsi que des références aux éléments du fichier \og .fxml \fg{} dans le code pour pouvoir les modifier hors de Scene Builder.
\subsection{Semaine 8: 10 avril - 14 avril}
Aide à l'implémentation du Workspace dans l'interface graphique principale avec déplacement et zoom.
\subsection{Semaine 9: 24 avril - 28 avril}
Continuation de l'implémentation du Workspace dans l'interface.
\subsection{Semaine 10: 1 mai - 5 mai}
Début d'élaboration de la zone \og Effet \fg{} de l'interface, recherche sur l'implémentation des effets en JavaFX ainsi que leur application sur divers éléments de la librairie (Image, texte, canvas).
\subsection{Semaine 11: 8 mai - 12 mai} 
Création de différents boutons permettant d'augmenter l'opacité, saturation et l'effet sepia. 
\subsection{Semaine 12: 15 mai - 19 mai}
Changements dans l'implémentation des effets, un \og Slider \fg{} JavaFX par effet incrémentable, permettant de modifier l'opacité, saturation, contraste et sepia sur un ou plusieurs calques. Implémentation d'une remise à zéro des effets. Début de réfection sur la sérialisation des effets.
\subsection{Semaine 13: 22 mai - 26 mai}
Ajout de l'effet de flou, ainsi que un bouton \og Tint \fg{} qui créé une nuance de couleur sur un calque. Implémentation de la sérialisation des effets et correction de l'ordre d'application des effets pour permettre leur application dans un ordre quelconque sans recréer tous les effets.
\subsection{Semaine 14: 29 mai - 2 juin}
Élaboration du rapport.




