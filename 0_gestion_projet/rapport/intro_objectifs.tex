\section{Introduction}
GEMMS est un éditeur d'images réalisé en Java en se basant sur les fonctionnalités graphiques de JavaFX 8. Il a été réalisé dans le cadre de la branche PRO (Projet de semestre) de la deuxième année d'informatique logicielle de la Haute-École d'Ingénierie et Gestion du canton de Vaud (HEIG-VD). 
Le programme a été élaboré sur une durée de 14 semaines aboutissant le 31 Mai 2017.

\section{Objectif}
L'application GEMMS a été conçue pour éditer des images de manière rapide, simple et intuitive. Le programme ne nécessite pas d'apprentissage particulier, comme certains programmes plus lourds comme Adobe Photoshop ou encore GIMP.

L'interface est propre, avec des icônes représentant les différents outils et actions possibles dans le programme en essayant de minimiser les menus déroulants surchargés. Des infobulles donnent une description concise de chaque outil lorsqu'on passe la souris dessus.

Bien que plus simple que les applications lourdes mentionnées ci-dessus, le concept de calques, très important dans l'édition d'image, est conservé. Certains éditeurs très basiques comme Paint sous Windows ne fournissent pas cette fonctionnalité. Les calques permettent de superposer des images, du texte, ou des canevas et de les déplacer, modifier ou effacer indépendamment les uns des autres. Lors de l'exportation du projet vers un format image, les calques sont aplatis et l'image est exportée en perdant ces informations.

Un projet GEMMS ne peut cependant pas uniquement être exporté en tant qu'image, mais également sauvegardé dans un fichier de projet \og.gemms \fg{}. Ce type de fichier peut être ouvert par l'application pour restaurer tout le projet en cours, recréant chaque calque ainsi que les transformations effectuées dessus.

Le programme doit pouvoir permettre d'effectuer les opérations classiques de manipulation d'image comme le dessin, l'ajout de texte, l'ajout d'image, le changement de taille, le rognage et bien d'autres. Une liste exhaustive des fonctionnalités implémentées ainsi que leur instructions d'utilisation peuvent être trouvées dans le manuel utilisateur.