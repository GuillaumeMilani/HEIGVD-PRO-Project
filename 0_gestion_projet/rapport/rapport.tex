% -----------------------------------------------------------------------
% --- DOCUMENT ---
% -----------------------------------------------------------------------

\documentclass[11pt, a4paper, french]{report}


% -----------------------------------------------------------------------
% --- PACKAGE ---
% -----------------------------------------------------------------------
\usepackage[french]{babel}

% Font
\usepackage[utf8]{inputenc}
\usepackage[T1]{fontenc}

% Marge du document
\usepackage[top=2cm, bottom=2cm, left=2cm, right=2cm]{geometry}

% Gérer les positionnement d'images
\usepackage{float}

% Import de fichier externe
\usepackage{graphicx}

% Mise en forme des URL
\usepackage{url}

% Informations sur un document compilé en PDF et les liens externes / internes
\usepackage{hyperref}

% Mise en page des tableaux
\usepackage{array}
\usepackage{tabularx}

% Espacement entre les lignes
\usepackage{setspace}

% Modifier la mise en page de l'abstract
\usepackage{abstract}

%pour utiliser \floatbarrier
%\usepackage{placeins}
%\usepackage{floatrow}



% -----------------------------------------------------------------------
% --- INFORMATION ET REGLES ---
% -----------------------------------------------------------------------
\selectlanguage{french} 

% Rajouter les numérotation pour les \paragraphe et \subparagraphe
\setcounter{secnumdepth}{4}
\setcounter{tocdepth}{4}

% Information sur le document
\hypersetup{							
pdfauthor = {
    Edward Ransome, 
    Guillaume Milani, 
    Mathieu Monteverde, 
    Michael Spierer,
    Sathiya Kirushnapillai},                    % Auteurs
pdftitle = {GEMMS - Edtieur d'image},           % Titre du document
pdfsubject = {Mémoire de Projet},               % Sujet
pdfkeywords = {Tag1, Tag2, Tag3, ...},          % Mots-clefs
pdfstartview={FitH}}                            % ajuste la page à la largueur de l'écran
%pdfcreator = {MikTeX},                         % Logiciel qui a crée le document
%pdfproducer = {}}                              % Société avec produit le logiciel


% -----------------------------------------------------------------------
% --- DEBUT DU DOCUMENT ---
% -----------------------------------------------------------------------
\begin{document}
    
    % Régler l'espacement entre les lignes
    \newcommand{\HRule}{\rule{\linewidth}{0.5mm}}
    
    % Page de garde
    \begin{titlepage}
    \begin{center}
        % Upper part of the page. The '~' is needed because only works if a paragraph has started.
        % \includegraphics[width=0.35\textwidth]{./logo}~\\[1cm]
        
        \textsc{\LARGE Haute Ecole d'Ingénierie et de Gestion du Canton de Vaud (HEIG-VD)}\\[1.5cm]
        
        \textsc{\Large }\\[0.5cm]
        
        % Title
        \HRule \\[0.4cm]
        
        {\huge \bfseries Projet de semestre (PRO)\\
            Editeur d'image GEMMS \\[0.4cm] }
        
        \HRule \\[1.5cm]
        
        % Author and supervisor
        \begin{minipage}{0.4\textwidth}
            \begin{flushleft} \large
                \emph{Auteur:}\\
                Edward \textsc{Ransome}\\
                Guillaume \textsc{Milani}\\
                Mathieu \textsc{Monteverde}\\
                Michael \textsc{Spierer}\\
                Sathiya \textsc{Kirushnapillai}
            \end{flushleft}
        \end{minipage}
        \begin{minipage}{0.4\textwidth}
            \begin{flushright} \large
                \emph{Client:} \\
                René \textsc{Rentsch}\\
                \emph{Référent:} \\
                René \textsc{Rentsch}\\
            \end{flushright}
        \end{minipage}
        
        \vfill
        
        % Bottom of the page
        {\large \today}
        
    \end{center}
\end{titlepage}

    
    % Page blanche
    \newpage
    ~
    %ne pas numéroter cette page
    \thispagestyle{empty}
    \newpage
    
    % \input{./abstract.tex}
    
    \tableofcontents
    \thispagestyle{empty}
    \setcounter{page}{0}
    %ne pas numéroter le sommaire
    
    \newpage
    
    %espacement entre les lignes d'un tableau
    \renewcommand{\arraystretch}{1.5}
    
    %====================== INCLUSION DES PARTIES ======================
    
    ~
    \thispagestyle{empty}
    %recommencer la numérotation des pages à "1"
    \setcounter{page}{0}
    \newpage
    
    \section{Introduction}
    
    \section{Conception et architecture}


    
    \section{Description technique}
    
    \section{Conclusion}

    
    \newpage
    
    %récupérer les citation avec "/footnotemark"
    \nocite{*}
    
    %choix du style de la biblio
    \bibliographystyle{plain}
    %inclusion de la biblio
    \bibliography{bibliographie.bib}
    %voir wiki pour plus d'information sur la syntaxe des entrées d'une bibliographie

\end{document}
