\section{Journal de travail - Michael Spierer}

\subsection{Semaine 1: 20 février - 24 février}
Nous avons fait un brainstorming des idées que nous avions, retenu les deux meilleures afin de les proposer
Constitution des groupes et choix du sujet. De nombreux sujets ont été proposés, la plupart ne faisant pas l'unanimité. Finalement, deux idées ont été retenues par le groupe: un programme de manipulation de graphes, et en premier lieu, un programme d'édition d'images.
\subsection{Semaine 2: 27 février - 3 mars}
Attribution des sujets de projet. Le projet de programme d'édition d'images a été accepté. Nous avons donc pu commencer la réflexion autour des fonctionnalités et la planification du projet.
\subsection{Semaine 3: 6 mars - 10 mars}
Discussion en groupe. Nous faisons le tri des fonctionnalités indispensables et utiles. Une fois celles-ci fixées, nous établissons le cahier des charges et la planification Gant qui va avec.
\subsection{Semaine 4: 13 mars - 17 mars}
Le professeur nous fourni un retour sur notre cahier des charges et notre planification. Pour plus de clarté, nous remplissons une nouvelle planification dans un format Excel. Nous nous mettons d'accord sur le fait de prendre deux semaines pour étudier la technologie JavaFX que nous utiliserons pour le projet et qu'aucun de nous ne connaît.
\subsection{Semaine 5: 20 mars - 24 mars}
Étude de JavaFX.
\subsection{Semaine 6: 27 mars - 31 mars}
Étude de JavaFX. Début de la mise en place de la structure de la classe Workspace avec la spécification des méthodes. Nous réalisons également que des difficultés sont à attendre pour la gestion des calques, de la sérialisation et des éléments issus de JavaFX en général.
\subsection{Semaine 7: 3 avril - 7 avril}
Implémentation du Workspace avec insertion et suppression de calques.
\subsection{Semaine 8: 10 avril - 14 avril}
Recherche de la gestion de calques avec Mathieu, on rencontre déjà des problèmes qui vont nous faire nous poser pas mal de questions, par exemple, un Node n'est pas sérialisable, donc on essaie de trouver des solutions à ce problème.
\subsection{Semaine 9: 24 avril - 28 avril}
Gestions de calques avec Mathieu. On a fait en sorte de rendre les calques sérialisables (on a ajouté des méthodes pour sérialiser et desérialiser avec l'aide de Sathiya car il avait déjà bosser sur la question de la sérialisation). On peut ajouter les calques au workspace.
\subsection{Semaine 10: 1 mai - 5 mai}
Implémentation des claques de texte.
\subsection{Semaine 11: 8 mai - 12 mai} 
Redimensionnement des calques, pivotement, deplacement d'un ou plusieurs calques. Il y a plusieurs manières d'aboutir aux effets escomptés, plusieurs choses rentrent en compte : vu que quasiment rien n'est sérialisable, il faut trouver comment, en plus de sérialiser les calques qui sont de base non-sérialisable, sérialiser les transformations.
\subsection{Semaine 12: 15 mai - 19 mai}
Toujours sur les redimensionnement de calques. Il y a de nouveau soucis : le mappage de coordonnées des tools (pas juste les transformations que j'implémente mais aussi brush etc) ne se fait pas bien --> On parvient à regler le soucis. De plus, les symetries m'ont donné du fil à retordre mais marchent parfaitement en fin de compte.
\subsection{Semaine 13: 22 mai - 26 mai}
Essaie de faire un bonus qui n'est pas dans le cahier des charges pour la réalisation d'un bucket (afin de remplir une zone d'une couleur en un clique). Le prototype marche bien sauf à partir du moment où il y a une transformation sur le calque. Je reévalue la situation, pas de temps à perdre sur cette fonctionnalité non demandée, je passe à la suite. Implémentation de l'alignement automatique d'un calque lors du drag.
\subsection{Semaine 14: 29 mai - 2 juin}
Finition du rapport et journal de bord.





