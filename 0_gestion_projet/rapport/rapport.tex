% -----------------------------------------------------------------------
% --- DOCUMENT ---
% -----------------------------------------------------------------------

\documentclass[11pt, a4paper, french]{article}



% -----------------------------------------------------------------------
% --- PACKAGE ---
% -----------------------------------------------------------------------
\usepackage[french]{babel}

% Font
\usepackage[utf8]{inputenc}
\usepackage[T1]{fontenc}

% Marge du document
\usepackage[top=3.5cm, 
            bottom=3cm, 
            left=2cm, 
            right=2cm, 
            footskip=1.5cm, 
            headheight=1.5cm, 
            headsep=0.9cm]{geometry}

% Gérer les positionnement d'images
\usepackage{float}

% Import de fichier externe
\usepackage{graphicx}

% Mise en forme des URL
\usepackage{url}

% Informations sur un document compilé en PDF et les liens externes / internes
\usepackage{hyperref}

% Mise en page des tableaux
\usepackage{array}
\usepackage{tabularx}

% Espacement entre les lignes
\usepackage{setspace}

% Modifier la mise en page de l'abstract
\usepackage{abstract}

\usepackage{titlesec}

\usepackage{fancyhdr}

%pour utiliser \floatbarrier
%\usepackage{placeins}
%\usepackage{floatrow}

% Pour voir les mages
%\usepackage{layout}
%\layout


% -----------------------------------------------------------------------
% --- INFORMATION SUR LE DOCUMENT
% -----------------------------------------------------------------------

% Information sur le document
\hypersetup{							
    pdfauthor = {
        Edward Ransome, 
        Guillaume Milani, 
        Mathieu Monteverde, 
        Michael Spierer,
        Sathiya Kirushnapillai},                    % Auteurs
    pdftitle = {GEMMS - Edtieur d'image},           % Titre du document
    pdfsubject = {Mémoire de Projet},               % Sujet
    pdfkeywords = {Tag1, Tag2, Tag3, ...},          % Mots-clefs
    pdfstartview={FitH}}                            % ajuste la page à la largueur de l'écran
%pdfcreator = {MikTeX},                             % Logiciel qui a crée le document
%pdfproducer = {}}                                  % Société avec produit le logiciel



% -----------------------------------------------------------------------
% --- EN-TETE ET PIED DE PAGE ---
% -----------------------------------------------------------------------
\pagestyle{fancy}
\fancyhf{} % Supprime les entetes et pieds de page existants

\fancyhead[L]{Projet de semestre\\}
\fancyhead[R]{IL - TIC - HEIG-VD \\ Printemps 2017}
\fancyfoot[L]{E. Ransome, G. Milani, S. Kirushnapillai, M. Spierer, M. Monteverde}
\fancyfoot[R]{\thepage{}}


\title{Éditeur d'images GEMMS \\ Cahier des charges}
\author{E. Ransome, G. Milani, S. Kirushnapillai, M. Spierer, M. Monteverde}
\date{Mars 2017}



% -----------------------------------------------------------------------
% --- DEBUT DU DOCUMENT ---
% -----------------------------------------------------------------------
\begin{document}
\selectlanguage{french} 

% Espacement entre les lignes
\newcommand{\HRule}{\rule{\linewidth}{0.5mm}}

% Page de garde
\begin{titlepage}
    \begin{center}
        % Upper part of the page. The '~' is needed because only works if a paragraph has started.
        % \includegraphics[width=0.35\textwidth]{./logo}~\\[1cm]
        
        \textsc{\LARGE Haute Ecole d'Ingénierie et de Gestion du Canton de Vaud (HEIG-VD)}\\[1.5cm]
        
        \textsc{\Large }\\[0.5cm]
        
        % Title
        \HRule \\[0.4cm]
        
        {\huge \bfseries Projet de semestre (PRO)\\
            Editeur d'image GEMMS \\[0.4cm] }
        
        \HRule \\[1.5cm]
        
        % Author and supervisor
        \begin{minipage}{0.4\textwidth}
            \begin{flushleft} \large
                \emph{Auteurs:}\\
                Guillaume \textsc{Milani}\\
                Edward \textsc{Ransome}\\
                Mathieu \textsc{Monteverde}\\
                Michael \textsc{Spierer}\\
                Sathiya \textsc{Kirushnapillai}
            \end{flushleft}
        \end{minipage}
        \begin{minipage}{0.4\textwidth}
            \begin{flushright} \large
                \emph{Client:} \\
                René \textsc{Rentsch}\\
                \emph{Référent:} \\
                René \textsc{Rentsch}\\
            \end{flushright}
        \end{minipage}
        
        \vfill
        
        % Bottom of the page
        {\large \today}
        
    \end{center}
\end{titlepage}

\newpage~


\thispagestyle{empty}
\newpage


% Table des matières
\tableofcontents
\thispagestyle{empty}
\newpage


% Espacement entre les lignes d'un tableau
\renewcommand{\arraystretch}{1.5}

% Config des pages
\setlength{\parskip}{1em}
\setlength{\parindent}{0pt}

\titlespacing\section{0pt}{12pt plus 4pt minus 2pt}{0pt plus 2pt minus 2pt}
\titlespacing\subsection{0pt}{12pt plus 4pt minus 2pt}{0pt plus 2pt minus 2pt}
\titlespacing\subsubsection{0pt}{12pt plus 4pt minus 2pt}{0pt plus 2pt minus 2pt}


~
\thispagestyle{empty}
% Recommencer la numérotation des pages à "1"
\setcounter{page}{0}
\newpage

%====================== INCLUSION DES PARTIES ======================

\section{Introduction}
GEMMS est un éditeur d'images réalisé en Java en se basant sur les fonctionnalités graphiques de JavaFX 8. Il a été réalisé dans le cadre de la branche PRO (Projet de semestre) de la deuxième année d'informatique logicielle de la Haute-École d'Ingénierie et Gestion du canton de Vaud (HEIG-VD). 
Le programme a été élaboré sur une durée de 14 semaines aboutissant le 31 Mai 2017.

\section{Objectifs}
L'application GEMMS a été conçue pour éditer des images de manière rapide, simple et intuitive. Le programme ne nécessite pas d'apprentissage particulier, comme certains programmes plus lourds comme Adobe Photoshop ou encore GIMP.

L'interface est propre, avec des icônes représentant les différents outils et actions possibles dans le programme en essayant de minimiser les menus déroulants surchargés. Des infobulles donnent une description concise de chaque outil lorsqu'on passe la souris dessus.

Bien que plus simple que les applications lourdes mentionnées ci-dessus, le concept de calques, très important dans l'édition d'image, est conservé. Certains éditeurs très basiques comme Paint sous Windows ne fournissent pas cette fonctionnalité. Les calques permettent de superposer des images, du texte, ou des canevas et de les déplacer, modifier ou effacer indépendamment les uns des autres. Lors de l'exportation du projet vers un format image, les calques sont aplatis et l'image est exportée en perdant ces informations.

Un projet GEMMS ne peut cependant pas uniquement être exporté en tant qu'image, mais également sauvegardé dans un fichier de projet \og.gemms \fg{}. Ce type de fichier peut être ouvert par l'application pour restaurer tout le projet en cours, recréant chaque calque ainsi que les transformations effectuées dessus.

Le programme doit pouvoir permettre d'effectuer les opérations classiques de manipulation d'image comme le dessin, l'ajout de texte, l'ajout d'image, le changement de taille, le rognage et bien d'autres. Une liste exhaustive des fonctionnalités implémentées ainsi que leur instructions d'utilisation peuvent être trouvées dans le manuel utilisateur.

\section{Conception \& Architecture}
\subsection{Technologies utilisées}
\subsubsection{Java 8}
Parmi les deux langages de haut niveau proposés pour l'élaboration de ce projet (Java ou C++), nous avons choisi Java pour sa portabilité, sa sécurité et ses performances. De plus, la dernière version de Java propose la librairie graphique JavaFX qui correspond en tout point à notre projet. Enfin, notre équipe est également plus habile à programmer à l'aide du langage Java.

\subsubsection{JavaFX 8}
JavaFX, successeur de Swing, est la librairie de création d'interface graphiques officielle de Java. La version 8, utilisée pour ce projet, ajoute de nouvelles fonctionnalités et est la plus récente version utilisable avec Scene Builder.

\begin{figure}[h]
    \caption{Architecture de JavaFX}
    \centering
    \includegraphics[width=\textwidth]{arch_javafx.png}
    \label{fig:arch_javafx}
\end{figure}

Comme vous pouvez le voir sur la figure \ref{fig:arch_javafx}, BLA BLA BLA A COMPLETER, A PARLER DE CSS ETC


\subsubsection{Scene Builder 8.3.0}
Scene Builder de Gluon permet de manipuler des objets JavaFX graphiquement et exporter ceux-ci dans un fichier .fxml interprétable par la librairie graphique. L'interface de base à été conçue lors de l'élaboration du cahier des charges pour présenter un exemple de l'interface de l'application finale. Plusieurs mock-ups ont été présentés et c'est sur ceux-cis que nous nous sommes basés pour construire, grâce à Scene Builder, une base d'interface sur la laquelle nous avons rajouté des composants et fonctionnalités tout au long de l'élaboration de l'application. La flexibilité de JavaFX permet d'ajouter des éléments via un fichier externe fxml mais aussi directement dans le code, ce que nous avons aussi utilisé.

\subsubsection{Maven}
Pour la compilation du projet et l'importation aisée de celui-ci dans un nouvel environnement de travail, nous avons utilisé l'outil Maven de Apache.
TODO TODO TODO TODO TODO COMPLETER

\subsubsection{Git}
Git est un logiciel de gestion de version utilisé pour permettre de stocker tous les fichiers du projet ainsi que toutes les modifications leur ayant été apportés depuis leur création. Pour chaque nouvelle fonctionnalité, nous avons procédé par la création d'une branche à partir de la branche principale (une version fonctionnelle du programme, contenant les fonctionnalités implémentées et testées). Ces nouvelles branches permettent de développer les fonctionnalités du programme indépendamment et de les ajouter à la branche principale une fois inspectées et testées par plusieurs membres de l'équipe.

\subsubsection{GitHub}
Github est un service web permettant de parcourir visuellement l'historique Git ainsi que de fournir des outils de gestion de Git. Notamment, pour chaque fonctionnalité ou chaque bug découvert, une "issue" (un problème) peut être ouverte et assignée à un ou plusieurs membres de l'équipe. Dès la fin de l'élaboration du planning de notre projet, des issues ont été assignées à chaque développeur. Celles-ci ont permis de mieux se fier au planning et toujours avoir en vue ce qu'il restait à implémenter.

\subsection{Comparaison de l'interface finale avec notre mock-up}
TODO TODO TODO TODO TODO Comparer les deux avec une image et parler un peu des changements éventuels

\section{Description technique}

Comme cité précédemment, notre application a été codé à l'aide de la librairie JavaFX. Ainsi, toute notre implémentation technique est basée sur cette dernière. 

\subsection{Structure}
JavaFX utilise des fichiers FXML pour séparer la logique de la vue TODO TODO TODO
ImageView, Canvas et Text
Controller

\subsection{Serialisation}
En Java, la sérialisation s'effectue à l'aide de l'interface \og Serializable \fg{}. Par conséquent, chaque classes de Java implémentant cette dernière telle que \og String \fg{}, peut être sérialisé et désérialisé à volonté. Cependant, la majorité des classes JavaFX n'implémente pas cette interface. En effet, cette librairie utilise grandement des mécanismes et des liaisons dynamiques tel que les listeners qui sont pour l'instant des sous-systèmes non-sérialisable. C'est pourquoi, JavaFX contient peu d'objet sérialisable.

Pour combler ce manque, nous devons nous même implémenter la sérialisation des classes JavaFX que nous sommes susceptible d'utiliser. 

\begin{figure}[h]
    \caption{Diagramme de la sérialisation simplifié}
    \centering
    \includegraphics[scale=0.6]{serialisation_diagram.png}
    \label{fig:seri_diag}
\end{figure}

Sur la figure \ref{fig:seri_diag}, nous pouvons voir un diagramme simplifié de l'implémentation de la sérialisation. Dans notre application, nous allons utilisé des classes de base telles que ImageView, Text, Canvas, Color, etc. Nous devons donc spécialiser ces classes afin qu'elles puissent implémenter l'interface \og Serializable \fg{}. Toutefois, certaines classes comme \og Color \fg{} ne sont malheureusement pas spécialisable. Il faut donc sérialiser les paramètres un par un à l'aide des accesseurs et mutateurs de cette dernière.

Étant donné que les classes JavaFX possèdent énormément de fonctionnalités, sérialiser l'entier de celles-ci nous demanderait beaucoup trop de temps. C'est pourquoi nous nous contentons uniquement des paramètres utilisés au sein du projet tel que la largeur, la hauteur, la position, etc. 

\lstinputlisting[language=Java, caption=Exemple de sérialisation]{./src/serialisation.java}

Bien que la sérialisation soit possible, ceci engendre des contraintes et des pertes de performances. Par exemple, les classes spécialisées ne peuvent plus étendre d'une classe commune et bénéficier de ses méthodes. De plus, les objets comme Canvas et ImageView devront sérialiser pixel par pixel, ce qui peut être long et volumineux selon la taille.

\subsection{Sauvegarde}
La sauvegarde d'un document utilise la sérialisation des objets. Comme mentionné précédemment, la sérialisation de certaines classes peut être volumineux. Ainsi, les données sont compressées dans le format GZIP.

\subsection{Workspace et liste des calques}
TODO TODO TODO

\subsection{Copier-coller}
TODO TODO TODO

\subsection{Historique}
Pour garder un historique de chaque action effectuée, on utilise la sérialisation des composants présentée précédemment. A la fin de chaque action modifiant l'espace de travail, une fonction va être appelée permettant de sauvegarder intégralement l'espace de travail courant et le placer sur une pile. A chaque détection de la commande Ctrl + Z, la sauvegarde sera chargée et la modification sera donc effacée. De même, à la détection de la commande Ctrl + Y, on va charger un espace de travail plus récent (s'il y en à un, c'est-à-dire si le Ctrl + Y était précédé d'un Ctrl + Z).

\subsection{Positionnement}
TODO TODO TODO 

\subsection{Outils}
\subsubsection{Pinceau}
\subsubsection{Gomme}
\subsubsection{Pipette}
\subsubsection{Modification de texte}
\subsubsection{Symétries}
\subsubsection{Déplacement}
\subsubsection{Rotation}
\subsubsection{Redimensionnement}
\subsubsection{Sélection}
\subsubsection{Rognage}

\subsection{Effets}
La section \og Effet \fg{} permet d'appliquer des effets colorimétriques, régler la transparence et ajouter un flou à un ou plusieurs calques. Chaque calque qui doit être modifié passe par une vérification: s'il n'a aucun effet appliqué, on lui applique trois effets: un ColorAdjust, un SepiaTone et un GaussianBlur permettant d'effectuer respectivement des ajustements de la couleur, un effet sépia, et un flou. Ces effets sont initialiement tous réglés pour n'effectuer aucun changement visuel. Ceci permet, lors d'un futur changement des effets, de simplement devoir faire varier les paramètres de ces effets. 

\subsubsection{Noir et blanc}
Le bouton \og B&W \fg{} à pour effet de régler la saturation de l'image à sa valeur minimale, donnant pour effet une image en nuances de gris.

\subsubsection{Tint}
Applique une teinture de la couleur sélectionnée à l'image. Ceci est fait en calculant une valeur de la teinte et en l'appliquant à l'effet ColorAdjust.

\subsubsection{Barres glissantes}
Les barres glissantes (\og Sliders \fg{} JavaFX) permettent d'ajuster tous les autres effets implémentés. La transparence règle directement un attribut \og Opacity \fg{} d'un noeud. La saturation, le contraste et la luminosité peuvent être ajustés en modifiant l'effet ColorAdjust. L'effet sépia et le flou ajustent respectivement l'effet SepiaTone et GaussianBlur.

\subsection{Remise à zéro}
Le bouton \og Reset \fg{} permet de remettre à zéro tous les effets sur tous les calques sélectionnés, les restaurant à leur état visuel initial.





\section{Conclusion}
Le programme fourni connaît certains problèmes de performances dû aux choix d'implémentation effectués. La taille des sauvegardes même après compression reste grande, ce qui implique que notre façon de garder un historique des changements est gourmand. Après chaque changement, on effectue une sauvegarde intégrale de l'espace de travail pour qu'il puisse être restauré par la suite. Ainsi, si l'on travaille sur des grandes images, on peut voir apparaître un délai après chaque action ou chaque utilisation d'un Ctrl + Z. En plus de ces ralentissements, la quantité de mémoire utilisée peut s'avérer problématique.

À notre avis, ceci est le plus grand défaut de notre application. Pour corriger ceci, l'implémentation de l'historique aurait pu être effectuée différemment: pour chaque action effectuée, il aurait été possible de sauvegarder un objet permettant d'effectuer l'action inverse. Par exemple, pour une symétrie horizontale, une implémentation possible aurait été d'empiler sur l'historique un objet effectuant une autre symétrie horizontale. Avec cette façon de faire, on évite de sauvegarder l'intégralité de l'espace de travail et on ne stocke qu'un objet très petit. Le problème de cette implémentation est sa difficulté d'implémentation. Pour chaque transformation, il doit être possible de coder son inverse et stocker les calques sur lesquelles il à été effectué. Avec un grand nombre d'actions possibles dans le programme cela représente une charge de travail considérable et, bien que plus performant, le temps nécessaire à son implémentation n'était simplement pas disponible.

Certaines améliorations du programme pourraient être effectuées. Notamment les fonctionnalités optionnelles que nous n'avons pas eu le temps d'implémenter. Par exemple, pouvoir paramétrer l'interface pour placer les outils, l'historique et l'affichage des calques à des endroits différents ou encore les cacher. Une autre fonctionnalité aurait été l'importation d'une collection d'images personnalisée pouvant être affiché dans l'interface et qui, avec un glisser-déposer, permettrait d'ajouter plus rapidement des images à un projet. Cette partie de l'interface pourrait contenir des images prédéfinies, comme des émoticônes susceptibles d'être ajoutés à un document. De manière générale, plus d'outils pourraient être ajoutés comme la sélection automatique, la sélection de forme quelconque, le pot de peinture ou plus d'outils de dessin. Avec l'interface modulaire de notre application, ajouter des outils à celle-ci est simple et ne risque pas de casser la mise en forme.

Pour ce qui concerne le travail de groupe, il a été difficile au début du projet de répartir des tâches pouvant être effectuées individuellement. Vu que le programme est composé entièrement d'une unique interface graphique, il existe beaucoup de dépendances entre les fonctionnalités. Travailler sans se marcher dessus et sans devoir attendre des fonctionnalités n'a pas toujours été facile. Notre planification des tâche initiale à cependant pris en compte ces dépendances et l'ordre d'implémentation des fonctionnalités à permis, dès l'élaboration de l'interface principale, d'éviter la plus part des conflits. Cependant, il s'est de temps en temps avéré qu'un changement de l'un des membres du groupe affecte les outils d'autres membres, notamment en ce qui concerne les systèmes de coordonnées.

Au final, le programme réalisé nous semble correspondre au cahier des charges proposé et, malgré certains problèmes de performance, est un outil qui nous restera utile en dehors du cadre de ce cours.



\newpage

%récupérer les citation avec "/footnotemark"
\nocite{*}

%choix du style de la biblio
\bibliographystyle{plain}
%inclusion de la biblio
\bibliography{bibliographie.bib}
%voir wiki pour plus d'information sur la syntaxe des entrées d'une bibliographie

\end{document}
