\newglossaryentry{workspace}{%
  name={Workspace},
  description={Workspace est une classe de l'application GEMMS représentant la zone de travail de l'utilisateur dans un Document GEMMS. Un objet contient notamment la liste de tous les calques utilisés dans le document}
}
\newglossaryentry{stage}{%
	name={Stage},
	description={Le Stage est le container haut-niveau d'une application JavaFX. Il doit contenir tout le contenu d'une fenêtre}
}

\newglossaryentry{scene}{%
	name={Scene},
	description={La Scene, dans une application JavaFX, est le container pour un Scene Graph. Chaque scène doit avoir un et un seul nœud racine du Scene Graph}
}
\newglossaryentry{scene_graphe}{%
	name={Scene Graph},
	description={Le Scene Graph (graphe de scène en français) est une structure en arbre qui garde une représentation interne des éléments graphiques de l'application}
}
\newglossaryentry{coloradjust}{%
name={ColorAdjust},
description={Effet JavaFX permettant d'ajuster la couleur d'un n\oe ud en modifiant son contraste, saturation, teinte et luminosité.}
}
\newglossaryentry{sepiatone}{%
name={SepiaTone},
description={Effet JavaFX permettant d'affecter à un n\oe ud un effet sépia d'intensité variable.}
}
\newglossaryentry{gaussianblur}{%
name={GaussianBlur},
description={Effet JavaFX permettant d'affecter à un n\oe ud un effet de flou gaussien d'intensité variable.}
}
\newglossaryentry{node}{name={Node},description={La classe Node est une classe de JavaFX. Il s'agit de la classe de base de tout élément présent dans le Scene Graph}}
\newglossaryentry{anchorpane}{name={AnchorPane},description={AnchorPane est une classe de JavaFX permettant (entre autres) de disposer des noeuds selon des coordonnées (x,y).}}
\newglossaryentry{canvas}{name={Canvas},description={Type de Node spécialisé permettant de dessiner des lignes, des formes et des éléments graphiques dans un canevas dans le Scene Graph.}}
\newglossaryentry{text}{name={Text},description={Type de Node spécialisé permettant de représenter un texte dans le Scene Graph.}}
\newglossaryentry{imageview}{name={ImageView},description={Type de Node personnalisé permettant de représenter une image dans le Scene Graph}}
\newglossaryentry{gemmscanvas}{name={GEMMSCanvas},description={Classe spécialisant la classe Canvas de JavaFX pour les besoins de l'application.}}
\newglossaryentry{gemmstext}{name={GEMMSText},description={Classe spécialisant la classe Text de JavaFX pour les besoins de l'application.}}
\newglossaryentry{gemmsimage}{name={GEMMSImage},description={Classe spécialisant la classe ImageView de JavaFX pour les besoins de l'application.}}
\newglossaryentry{serialisation}{name={Sérialisation},description={La sérialisation, c'est rendre un objet persistant afin de pouvoir le stocker ou l'échanger de manière de textuelle}}
\newglossaryentry{color}{name={Color},description={Classe JavaFX permettant de représenter une couleur.}}
\newglossaryentry{stackpane}{name={Stackpane},description={Classe JavaFX permettant de positionner des n\oe uds sur une pile de l'arrière vers l'avant.}}
\newglossaryentry{listview}{name={ListView},description={Classe JavaFX permettant de représenter une liste d'éléments dans une liste verticale ou horizontale de cellules correspondantes.}}
\newglossaryentry{layerlist}{name={LayerList},description={LayerList est une classe implémentée pour l'application GEMMS qui est utilisée pour afficher les calques d'un document sous la forme d'une liste verticale de cellules.}}
\newglossaryentry{observablelist}{name={ObservableList},description={Interface JavaFX représentant le comportement d'une liste pouvant être observée dans le but d'être notifié des changements qui y sont appliqués.}}