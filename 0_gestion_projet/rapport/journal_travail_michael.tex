\section{Journal de travail - Michael Spierer}

\subsection{Semaine 1: 20 février - 24 février}
Nous avons fait un brainstorming afin de trouver des idées, retenu les deux meilleures afin de les proposer pour le projet.
\subsection{Semaine 2: 27 février - 3 mars}
Notre idée de faire un éditeur d'images a été accepté et nous avons pu commencer à trouver quelles fonctionnalités nous voulons avoir et à commencer à les planifier. 
\subsection{Semaine 3: 6 mars - 10 mars}
Cahier des charges et planification Gant.
\subsection{Semaine 4: 13 mars - 17 mars}
Planification dans un format Excel.
\subsection{Semaine 5: 20 mars - 24 mars}
Étude de JavaFX, tests de certaines fonctionnalitées de JavaFX, tutoriels de JavaFX.
\subsection{Semaine 6: 27 mars - 31 mars}
Étude de JavaFX, tests de certaines fonctionnalitées afin de jeter un coup d'oeil aux fonctionnalitées que je vais devoir implémenter.
\subsection{Semaine 7: 3 avril - 7 avril}
Début d'implémentation du Workspace.
\subsection{Semaine 8: 10 avril - 14 avril}
Recherche de la gestion de calques avec Mathieu, on rencontre déjà des problèmes qui vont nous faire nous poser pas mal de questions, par exemple, un Node n'est pas sérialisable, donc on essaie de trouver des solutions à ce problème.
\subsection{Semaine 9: 24 avril - 28 avril}
Gestions de calques avec Mathieu. On a fait en sorte de rendre les calques sérialisables (on a ajouté des méthodes pour sérialiser et desérialiser avec l'aide de Sathiya car il avait déjà bosser sur la question de la sérialisation). On peut ajouter les calques au workspace.
\subsection{Semaine 10: 1 mai - 5 mai}
Implémentation des claques de texte.
\subsection{Semaine 11: 8 mai - 12 mai} 
Redimensionnement des calques, pivotement, deplacement d'un ou plusieurs calques. Il y a plusieurs manières d'aboutir aux effets escomptés, plusieurs choses rentrent en compte : vu que quasiment rien n'est sérialisable, il faut trouver comment sérialiser les transformations, en plus de sérialiser les calques qui sont de base non-sérialisable.
\subsection{Semaine 12: 15 mai - 19 mai}
Toujours sur les redimensionnement de calques. Il y a de nouveau soucis : le mappage de coordonnées des tools (pas juste les transformations que j'implémente mais aussi brush etc) ne se fait pas bien --> On parvient à regler le soucis. De plus, les symetries m'ont donné du fil à retordre mais marchent parfaitement en fin de compte.
\subsection{Semaine 13: 22 mai - 26 mai}
Essaie de faire un bonus qui n'est pas dans le cahier des charges pour la réalisation d'un bucket (afin de remplir une zone d'une couleur en un clique). Le prototype marche bien sauf à partir du moment où il y a une transformation sur le calque. Je reévalue la situation, pas de temps à perdre sur cette fonctionnalité non demandée, je passe à la suite. Implémentation de l'alignement automatique d'un calque lors du drag.
\subsection{Semaine 14: 29 mai - 2 juin}
Rapport et journal de bord.





