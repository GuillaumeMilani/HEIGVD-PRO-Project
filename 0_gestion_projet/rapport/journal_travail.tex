$
% -----------------------------------------------------------------------
% --- DOCUMENT ---
% -----------------------------------------------------------------------

\documentclass[11pt, a4paper, french]{article}



% -----------------------------------------------------------------------
% --- PACKAGE ---
% -----------------------------------------------------------------------
\usepackage[french]{babel}

% Font
\usepackage[utf8]{inputenc}
\usepackage[T1]{fontenc}

% Marge du document
\usepackage[top=3.5cm, 
bottom=3cm, 
left=2cm, 
right=2cm, 
footskip=1.5cm, 
headheight=1.5cm, 
headsep=0.9cm]{geometry}

% Gérer les positionnement d'images
\usepackage{float}

% Import de fichier externe
\usepackage{graphicx}

% Mise en forme des URL
\usepackage{url}

% Informations sur un document compilé en PDF et les liens externes / internes
\usepackage{hyperref}

% Mise en page des tableaux
\usepackage{array}
\usepackage{tabularx}

% Espacement entre les lignes
\usepackage{setspace}

% Modifier la mise en page de l'abstract
\usepackage{abstract}

\usepackage{titlesec}

% Pour les entêtes
\usepackage{fancyhdr}

% Pour l'import de code
\usepackage{listings}

% Utilisation de couleur
\usepackage{color}

%pour utiliser \floatbarrier
%\usepackage{placeins}
%\usepackage{floatrow}

% Pour voir les mages
%\usepackage{layout}
%\layout



% -----------------------------------------------------------------------
% --- Couleurs définies
% -----------------------------------------------------------------------
\definecolor{pblue}{rgb}{0.13,0.13,1}
\definecolor{pgreen}{rgb}{0,0.5,0}
\definecolor{pred}{rgb}{0.9,0,0}
\definecolor{pgrey}{rgb}{0.46,0.45,0.48}
\definecolor{backcolour}{gray}{0.95}


% -----------------------------------------------------------------------
% --- Listing de code
% -----------------------------------------------------------------------
\lstset{language=Java,
    backgroundcolor=\color{backcolour},
    showspaces=false,
    showtabs=false,
    breaklines=true,
    xleftmargin=0.5cm,
    framexleftmargin=0.5cm,
    framextopmargin=6pt,
    framexbottommargin=6pt, 
    frame=tb, 
    framerule=0pt,
    showstringspaces=false,
    breakatwhitespace=true,
    commentstyle=\color{pgreen},
    keywordstyle=\color{pblue},
    stringstyle=\color{pred},
    basicstyle=\ttfamily,
    moredelim=[is][\textcolor{pgrey}]{\%\%}{\%\%}
}



% -----------------------------------------------------------------------
% --- INFORMATION SUR LE DOCUMENT
% -----------------------------------------------------------------------

% Information sur le document
\hypersetup{							
    pdfauthor = {
        Edward Ransome, 
        Guillaume Milani, 
        Mathieu Monteverde, 
        Michael Spierer,
        Sathiya Kirushnapillai},                    % Auteurs
    pdftitle = {GEMMS - Edtieur d'image},           % Titre du document
    pdfsubject = {Mémoire de Projet},               % Sujet
    pdfkeywords = {Tag1, Tag2, Tag3, ...},          % Mots-clefs
    pdfstartview={FitH}}                            % ajuste la page à la largueur de l'écran
%pdfcreator = {MikTeX},                             % Logiciel qui a crée le document
%pdfproducer = {}}                                  % Société avec produit le logiciel



% -----------------------------------------------------------------------
% --- EN-TETE ET PIED DE PAGE ---
% -----------------------------------------------------------------------
\pagestyle{fancy}
\fancyhf{} % Supprime les entetes et pieds de page existants

\fancyhead[L]{Projet de semestre\\}
\fancyhead[R]{IL - TIC - HEIG-VD \\ Printemps 2017}
\fancyfoot[L]{E. Ransome, G. Milani, S. Kirushnapillai, M. Spierer, M. Monteverde}
\fancyfoot[R]{\thepage{}}


\title{Éditeur d'images GEMMS \\ Cahier des charges}
\author{E. Ransome, G. Milani, S. Kirushnapillai, M. Spierer, M. Monteverde}
\date{Mars 2017}



% -----------------------------------------------------------------------
% --- DEBUT DU DOCUMENT ---
% -----------------------------------------------------------------------
\begin{document}
    \selectlanguage{french}
    \graphicspath{ {img/} }
    
    % Espacement entre les lignes
    \newcommand{\HRule}{\rule{\linewidth}{0.5mm}}
    
    % Page de garde
    \begin{titlepage}
    \begin{center}
        % Upper part of the page. The '~' is needed because only works if a paragraph has started.
        % \includegraphics[width=0.35\textwidth]{./logo}~\\[1cm]
        
        \textsc{\LARGE Haute Ecole d'Ingénierie et de Gestion du Canton de Vaud (HEIG-VD)}\\[1.5cm]
        
        \textsc{\Large }\\[0.5cm]
        
        % Title
        \HRule \\[0.4cm]
        
        {\huge \bfseries Projet de semestre (PRO)\\
            Editeur d'image GEMMS \\[0.4cm] }
        
        \HRule \\[1.5cm]
        
        % Author and supervisor
        \begin{minipage}{0.4\textwidth}
            \begin{flushleft} \large
                \emph{Auteur:}\\
                Edward \textsc{Ransome}\\
                Guillaume \textsc{Milani}\\
                Mathieu \textsc{Monteverde}\\
                Michael \textsc{Spierer}\\
                Sathiya \textsc{Kirushnapillai}
            \end{flushleft}
        \end{minipage}
        \begin{minipage}{0.4\textwidth}
            \begin{flushright} \large
                \emph{Client:} \\
                René \textsc{Rentsch}\\
                \emph{Référent:} \\
                René \textsc{Rentsch}\\
            \end{flushright}
        \end{minipage}
        
        \vfill
        
        % Bottom of the page
        {\large \today}
        
    \end{center}
\end{titlepage}

    \newpage~
    
    

    
    % Espacement entre les lignes d'un tableau
    \renewcommand{\arraystretch}{1.5}
    
    % Config des pages
    \setlength{\parskip}{1em}
    \setlength{\parindent}{0pt}
    
    \titlespacing\section{0pt}{12pt plus 4pt minus 2pt}{0pt plus 2pt minus 2pt}
    \titlespacing\subsection{0pt}{12pt plus 4pt minus 2pt}{0pt plus 2pt minus 2pt}
    \titlespacing\subsubsection{0pt}{12pt plus 4pt minus 2pt}{0pt plus 2pt minus 2pt}
    
    
    ~
    \thispagestyle{empty}
    % Recommencer la numérotation des pages à "1"
    \setcounter{page}{0}
    \newpage
	
	\section{Journal de travail - Edward Ransome}

\subsection{Semaine 1 20 février - 24 février}
Création du groupe et recherche d'idées. Discussion et proposition des deux principaux sujets au professeur, un éditeur d'images et un programme de manipulation de graphes.
\subsection{Semaine 2 27 février - 3 mars}
Programme d'édition d'images accepté, début d'élaboration du cahier des charges. Discussion sur le fonctionnement, l'architecture, les fonctionnalités voulues ainsi que des mock-ups d'interface. Travail simultané de tout le groupe.
\subsection{Semaine 3 6 mars - 10 mars}
Fin de rédaction du cahier des charges, avec mock-ups finaux et une liste des fonctionnalités indispensables et optionnelles. Rendu de celui-ci ainsi qu'un planning du travail sous forme de diagramme de Gant.
\subsection{Semaine 4 13 mars - 17 mars}
Retour sur le cahier des charges, pas de problèmes majeurs. Planification rédigée sous forme d'un tableau Excel pour faciliter la compréhension et mieux voir le travail effectué à chaque membre du groupe. Début du travail individuel selon le planning. 
\subsection{Semaine 5 20 mars - 24 mars}
Étude de JavaFX. Tutoriel JavaFX 8 de Code Makery effectué, qui comprends une introduction à Scene Builder, création de fenêtres, effets et autres fonctionnalités de cette librairie. Étude de la documentation Oracle de JavaFX.
\subsection{Semaine 6 27 mars - 31 mars}
Début de création de l'interface avec Guillaume Milani en se basant sur les mock-ups élaborés pour le cahier des charges.
\subsection{Semaine 7 3 avril - 7 avril}
Fin de l'interface Scene Builder avec contrôleurs des boutons dans le code ainsi que des références aux éléments du fichier \og .fxml \fg{} dans le code pour pouvoir les modifier hors de Scene Builder.
\subsection{Semaine 8 10 avril - 14 avril}
Aide à l'implémentation du Workspace dans l'interface graphique principale avec déplacement et zoom.
\subsection{Semaine 9 24 avril - 28 avril}
Continuation de l'implémentation du Workspace dans l'interface.
\subsection{Semaine 10 1 mai - 5 mai}
Début d'élaboration de la zone \og Effet \fg{} de l'interface, recherche sur l'implémentation des effets en JavaFX ainsi que leur application sur divers éléments de la librairie (Image, texte, canvas).
\subsection{Semaine 11 8 mai - 12 mai} 
Création de différents boutons permettant d'augmenter l'opacité, saturation et l'effet sepia. 
\subsection{Semaine 12 15 mai - 19 mai}
Changements dans l'implémentation des effets, un \og Slider \fg{} JavaFX par effet incrémentable, permettant de modifier l'opacité, saturation, contraste et sepia sur un ou plusieurs calques. Implémentation d'une remise à zéro des effets. Début de réfection sur la sérialisation des effets.
\subsection{Semaine 13 22 mai - 26 mai}
Ajout de l'effet de flou, ainsi que un bouton \og Tint \fg{} qui créé une nuance de couleur sur un calque. Implémentation de la sérialisation des effets et correction de l'ordre d'application des effets pour permettre leur application dans un ordre quelconque sans recréer tous les effets.
\subsection{Semaine 14 29 mai - 2 juin}
Finition du rapport et du manuel utilisateur.





    \subsection{Journal de travail - Mathieu Monteverde}

\subsubsection{Semaine 1 20 février - 24 février}
\subsubsection{Semaine 2 27 février - 3 mars}
\subsubsection{Semaine 3 6 mars - 10 mars}
\subsubsection{Semaine 4 13 mars - 17 mars}
\subsubsection{Semaine 5 20 mars - 24 mars}
\subsubsection{Semaine 6 27 mars - 31 mars}
\subsubsection{Semaine 7 3 avril - 7 avril}
\subsubsection{Semaine 8 10 avril - 14 avril}
\subsubsection{Semaine 9 24 avril - 28 avril}
\subsubsection{Semaine 10 1 mai - 5 mai}
\subsubsection{Semaine 11 8 mai - 12 mai}
\subsubsection{Semaine 12 15 mai - 19 mai}
\subsubsection{Semaine 13 22 mai - 26 mai}
\subsubsection{Semaine 14 29 mai - 2 juin}




    
    \section{Journal de travail - Sathiya Kirushnapillai}

\subsection{Semaine 1 20 février - 24 février}
Constitution des groupes et choix du sujet. De nombreux sujets ont été proposés, la plupart ne faisant pas l'unanimité. Finalement, deux idées ont fait été retenues par le groupe: un programme de manipulation de graphes, et en premier lieu, un programme d'édition d'images.
\subsection{Semaine 2 27 février - 3 mars}
Attribution des sujets de projet. Le projet de programme d'édition d'images a été accepté. Nous avons donc pu commencer la réflexion autour des fonctionnalités et la planification du projet.
\subsection{Semaine 3 6 mars - 10 mars}
Discussion en groupe. Nous faisons le tri des fonctionnalités indispensables et utiles. Une fois celles-ci fixées, nous établissons le cahier des charges et la planification Gant qui va avec.
\subsection{Semaine 4 13 mars - 17 mars}
Le professeur nous fourni un retour sur notre cahier des charges et notre planification. Pour plus de clarté, nous remplissons une nouvelle planification dans un format Excel. Nous nous mettons d'accord sur le fait de prendre deux semaines pour étudier la technologie JavaFX que nous utiliserons pour le projet et qu'aucun de nous ne connaît.
\subsection{Semaine 5 20 mars - 24 mars}
Étude de JavaFX.
\subsection{Semaine 6 27 mars - 31 mars}
Étude de JavaFX. Début de la mise en place de la structure de la classe Workspace avec la spécification des méthodes. Nous réalisons également que des difficultés sont à attendre pour la gestion des calques, de la sérialisation et des éléments issus de JavaFX en général.
\subsection{Semaine 7 3 avril - 7 avril}
Implémentation du Workspace avec insertion et suppression de calques.
\subsection{Semaine 8 10 avril - 14 avril}
Implémentation de la gestion du Workspace pour le déplacement et le zoom de l'utilisateur dans l'interface. Premières recherches pour la gestion de calques au moyen d'une ListView de JavaFX.
\subsection{Semaine 9 24 avril - 28 avril}
Ajout des prototypes recherchés en semaine 9 au reste du projet. Le Workspace permet maintenant d'ajouter des calques, de les supprimer, et de se déplacer dans l'interface. 
\subsection{Semaine 10 1 mai - 5 mai}
Des changements ont eu lieu pour le Workspace. De mon côté, il faut encore améliorer la gestion des calques.
On se rend compte que l'élément ListView de JavaFX, qui pourrait pourtant de fonctionner parfaitement pour l'affichage des calques, ne permet de pas de changer l'ordre d'affichage.
\subsection{Semaine 11 8 mai - 12 mai} 
Il va falloir trouver une solution pour la gestion des calques. Le module étant cependant fonctionnel, on se charge des autres fonctionnalités. Début de la réalisation du pinceau et de la gomme.
\subsection{Semaine 12 15 mai - 19 mai}
Remplacement de la ListView de gestion des calques JavaFX par un composant fait main pour pouvoir répondre aux besoins de l'application. Les outils pinceaux, gommes et pipette sont fonctionnels. Beaucoup d'élément ont pu être ajoutés cette semaine. La gestion de la couleur, le paramétrage des outils (taille du pinceau et de la gomme, édition de textes et autres).
\subsection{Semaine 13 22 mai - 26 mai}
Il y a eu beaucoup de problèmes avec les transformations de calques (rotation, taille, symétrie, etc.) Mais après beaucoup d'efforts, les problèmes ont pu être résolu. Maintenant, il s'agit de peaufiner l'interface (espacement, ergonomie, retour visuels pour l'utilisateur, etc.),
\subsection{Semaine 14 29 mai - 2 juin}
Finition du rapport et du manuel utilisateur.





    \section{Journal de travail - Michael Spierer}

\subsection{Semaine 1: 20 février - 24 février}
Nous avons constitué notre groupe et fait un brainstorming afin de trouver des idées. Nous retenons les deux meilleures afin de les proposer pour le projet.
\subsection{Semaine 2: 27 février - 3 mars}
Notre idée de faire un éditeur d'images a été accepté et nous avons pu commencer à écrire le cahier des charges et à trouver quelles fonctionnalités nous voulons avoir.
\subsection{Semaine 3: 6 mars - 10 mars}
Cahier des charges final et planification Gantt. Cette dernière prend pas mal de temps car un editeur d'image, d'après les conseils de Mr Rentsch aux vues des dernières années, amène beaucoup d'interdépendance au sein d'un groupe.
\subsection{Semaine 4: 13 mars - 17 mars}
Notre Gantt étant trop complexe niveau compréhension, on nous demande de le refaire sur Excel afin de le rendre plus claire. 
\subsection{Semaine 5: 20 mars - 24 mars}
Étude de JavaFX, tests de certaines fonctionnalitées de JavaFX, tutoriels de base de JavaFX.
\subsection{Semaine 6: 27 mars - 31 mars}
Étude de JavaFX, tests de certaines fonctionnalitées afin de jeter un coup d'oeil aux fonctionnalitées que je vais devoir implémenter par la suite.
\subsection{Semaine 7: 3 avril - 7 avril}
Début d'implémentation du Workspace.
\subsection{Semaine 8: 10 avril - 14 avril}
Recherche de la gestion de calques avec Mathieu, on rencontre déjà des problèmes qui vont nous faire nous poser pas mal de questions, par exemple, un Node n'est pas sérialisable, donc on essaie de trouver des solutions à ce problème.
\subsection{Semaine 9: 24 avril - 28 avril}
Gestions de calques avec Mathieu. On a fait en sorte de rendre les calques ( n\oe uds) sérialisables (on a ajouté des méthodes pour sérialiser et desérialiser avec l'aide de Sathiya car il avait déjà bossé sur la question de la sérialisation). On peut ajouter nos calques personnels (GEMMSImage par exemple) au workspace.
\subsection{Semaine 10: 1 mai - 5 mai}
Implémentation des calques de texte (GEMMSText) et leurs outils de modifications,(taille, etc.).
\subsection{Semaine 11: 8 mai - 12 mai} 
Redimensionnement des calques, pivotement, deplacement d'un ou plusieurs calques. Il y a plusieurs manières d'aboutir aux effets escomptés, plusieurs choses rentrent en compte : vu que quasiment rien n'est sérialisable, il faut trouver comment sérialiser les transformations, en plus de sérialiser les calques qui sont de base non-sérialisable.
\subsection{Semaine 12: 15 mai - 19 mai}
Toujours sur les outils de transformations de n\oe uds. Il y a de nouveau soucis : le mappage de coordonnées des tools (pas juste les transformations que j'implémente mais aussi brush etc) ne se fait pas bien. Pour finir, on parvient à regler le soucis. De plus, les symetries m'ont donné du fil à retordre mais marchent parfaitement en fin de compte. Ces symetries ne sont pas extremement dure niveau code mais on amené pas mal de problème interne, par exemple niveau sérialisation ou transformation.
\subsection{Semaine 13: 22 mai - 26 mai}
Implémentation de l'alignement automatique avec effet aimant d'un calque lors de son déplacement.
Essaie de faire un bonus qui n'est pas dans le cahier des charges pour la réalisation d'un bucket (afin de remplir une zone d'une couleur en un clique). Le prototype marche bien sauf à partir du moment où il y a une transformation sur le calque. Je reévalue la situation, pas de temps à perdre sur cette fonctionnalité non demandée, je passe à la suite. Implémentation de l'alignement automatique d'un calque lors du drag.
\subsection{Semaine 14: 29 mai - 2 juin}
Rapport et journal de bord.








    

\end{document}


